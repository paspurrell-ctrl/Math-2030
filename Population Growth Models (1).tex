\documentclass{article}
\usepackage{graphicx} % Required for inserting images
\usepackage{url} 
\title{Math 2030 Project 2}
\author{Paige Spurrell}
\date{February 2026}
\newpage
\begin{document}
\maketitle
\newpage
\tableofcontents
\newpage
\section{My Question}
What are the advantages and limitations of different population growth models?
\section{introduction}
My project will be discussing population growth models like logistic, exponential and linear. I will discuss how they differ as well as the  limitations of each model using multiple different articles to expand my research.
\section{Logistic, Exponential and Linear Population Growth}
In my final paper I will discuss where the logistic growth model is appropriate to use as well as mechanics of the models, explaining the S-shaped curve and what each part of the generalized equation means:
\begin{equation}
\frac{dN}{dt} = rN\left(1 - \frac{N}{K}\right)
\end{equation}
My paper will also do the same for exponential growth and discuss why this model creates a J-shape instead of an S-shaped like the logistic model. Again  will discuss the equation and what each part represents
\begin{equation}
\frac{dN}{dt} = rN
\end{equation}
\begin{center}
\bf or  
\end{center}
\begin{equation}
P(t) = P_0 e^{rt}
\end{equation}
I will also touch on linear population growth, but since its more straight forward I will not be going into depth with the explanation. A simple compare and contrast to the other two models as well as a brief explanation on the straight line model and the equation will be sufficient
\begin{equation}
y = mx + b
\end{equation}

\end{document}